\documentclass[
  captions=tableheading,  % Tabellenüberschriften
  titlepage=firstiscover, % Titelseite ist Deckblatt
]{scrartcl}

% Paket float verbessern
\usepackage{scrhack}

% Warnung, falls nochmal kompiliert werden muss
\usepackage[aux]{rerunfilecheck}

% unverzichtbare Mathe-Befehle
\usepackage{amsmath}
% viele Mathe-Symbole
\usepackage{amssymb}
% Erweiterungen für amsmath
\usepackage{mathtools}

\usepackage{amsthm}


% Fonteinstellungen
\usepackage{fontspec}
% Latin Modern Fonts werden automatisch geladen
% Alternativ zum Beispiel:
%\setromanfont{Libertinus Serif}
%\setsansfont{Libertinus Sans}
%\setmonofont{Libertinus Mono}

% Wenn man andere Schriftarten gesetzt hat,
% sollte man das Seiten-Layout neu berechnen lassen
\recalctypearea{}

% deutsche Spracheinstellungen
\usepackage[ngerman]{babel}

\usepackage{thmtools}

\usepackage[
  math-style=ISO,    % ┐
  bold-style=ISO,    % │
  sans-style=italic, % │ ISO-Standard folgen
  nabla=upright,     % │
  partial=upright,   % ┘
  warnings-off={           % ┐
    mathtools-colon,       % │ unnötige Warnungen ausschalten
    mathtools-overbracket, % │
  },                       % ┘
]{unicode-math}

% traditionelle Fonts für Mathematik
\setmathfont{Latin Modern Math}
% Alternativ zum Beispiel:
%\setmathfont{Libertinus Math}

\setmathfont{XITS Math}[range={scr, bfscr}]
\setmathfont{XITS Math}[range={cal, bfcal}, StylisticSet=1]

% Zahlen und Einheiten
\usepackage[
  locale=DE,                   % deutsche Einstellungen
  separate-uncertainty=true,   % immer Unsicherheit mit \pm
  per-mode=symbol-or-fraction, % / in inline math, fraction in display math
]{siunitx}

% chemische Formeln
\usepackage[
  version=4,
  math-greek=default, % ┐ mit unicode-math zusammenarbeiten
  text-greek=default, % ┘
]{mhchem}

% richtige Anführungszeichen
\usepackage[autostyle]{csquotes}

% schöne Brüche im Text
\usepackage{xfrac}

% Standardplatzierung für Floats einstellen
\usepackage{float}
\floatplacement{figure}{htbp}
\floatplacement{table}{htbp}

% Floats innerhalb einer Section halten
\usepackage[
  section, % Floats innerhalb der Section halten
  below,   % unterhalb der Section aber auf der selben Seite ist ok
]{placeins}

% Seite drehen für breite Tabellen: landscape Umgebung
\usepackage{pdflscape}

% Captions schöner machen.
\usepackage[
  labelfont=bf,        % Tabelle x: Abbildung y: ist jetzt fett
  font=small,          % Schrift etwas kleiner als Dokument
  width=0.9\textwidth, % maximale Breite einer Caption schmaler
]{caption}
% subfigure, subtable, subref
\usepackage{subcaption}

% Grafiken können eingebunden werden
\usepackage{graphicx}

% schöne Tabellen
\usepackage{booktabs}

% Verbesserungen am Schriftbild
\usepackage{microtype}


\usepackage{listings}

% Hyperlinks im Dokument
\usepackage[
  german,
  unicode,        % Unicode in PDF-Attributen erlauben
  pdfusetitle,    % Titel, Autoren und Datum als PDF-Attribute
  pdfcreator={},  % ┐ PDF-Attribute säubern
  pdfproducer={}, % ┘
]{hyperref}
% erweiterte Bookmarks im PDF
\usepackage{bookmark}

% Trennung von Wörtern mit Strichen
\usepackage[shortcuts]{extdash}

\usepackage{cleveref}

\author{%
  Yanick Sebastian Kind\\%
  \href{mailto:yanick.kind@udo.edu}{yanick.kind@udo.edu}%
}

\title{Zusammenfassung zur Vorlesung Gruppentheorie in der Physik I}
\date{03.03.2023}

\begin{document}

\maketitle

\tableofcontents

\theoremstyle{remark}
\newtheorem{df}{Definition}
\listoftheorems


\newpage
\section{Ergänzen}
\begin{itemize}
  \item Iso/Homomorphismus
  \item Permutationsgruppe
  \item triviale Darstellung als Isomorphismus
  \item vll. noch Kristallstruktur und Blochtheorem 
  \item wichtigen Liegruppen und dessen Mannigfaltigkeit
  \item Sachen von Henry
  \item Mannigfaltigkeiten verbessern evtl. Bilder
\end{itemize}
\section{Abstrakte Gruppentheorie}
\subsection{Definition: Gruppe}
  Eine Menge $\symcal{G} = \{A_2,A_3,...\}$ bildet eine Gruppe, wenn mit einer Gruppenverknüpfung $*$ folgende vier Eigenschaften erfüllt sind:
  \begin{enumerate}
    \item \textbf{Abgeschlossenheit}: Mit $A_i, A_j \in \symcal{G}$ folgt $A_i * A_j = A_k \in \symcal{G}$, d.h. die Verknüpfung zweier 
    Elemente ergibt wieder ein Element der Gruppe.
    \item \textbf{Assoziativität}: Es gilt mit $A_i, A_j, A_k \in \symcal{G}$, dass $(A_i * A_j) * A_k = A_i * (A_j*A_k)$.
    \item \textbf{Neutrale Element}: Es exestiert ein eindeutiges Element $E\in \symcal{G}$ mit $E * A_i = A_i * E  = A_i$.
    \item \textbf{Inverse Element}: Zu jedem Element $A_i \in \symcal{G}$ exestiert ein eindeutiges inverses Element $A_i^{-1}$,
      so dass $A_i^{-1} * A_i = A_i * A_i^{-1} = E$ gilt.
  \end{enumerate}
\subsubsection{Endliche Gruppe}
Eine Gruppe mit einer endlichen Anzahl an Elementen heißt endliche Gruppe.
Eine Gruppe $\symcal{G} = \{ E, A_2, \ldots, A_h \} $ ist eine endliche Gruppe der Ordnung $h$.
Man schreibt auch $|\symcal{G}| = h$.
\footnote{Im Folgenden wird das Symbol der Verknüpfung und die Angabe, dass ein Element ein Element
einer Gruppe ist, weggelassen, sofern es eindeutig ist.}
\subsection{Multiplikationstabelle}
Die Multiplikationstabelle gibt einfach an, welche Verknüpfungen welches Gruppenelement ergeben.
Bsp. Symmetrische Gruppe $S_3$:
\[
    \begin{tabular}{>{$}l<{$}|*{6}{>{$}l<{$}}}
    ~   & e   & a   & a^2 & b   & c   & d   \\
    \hline\vrule height 12pt width 0pt
    e   & e   & a   & a^2 & b   & c   & d   \\
    a   & a   & a^2 & e   & c   & d   & b   \\
    a^2 & a^2 & e   & a   & d   & b   & c   \\
    b   & b   & d   & c   & e   & a^2 & a   \\
    c   & c   & b   & d   & a   & e   & a^2 \\
    d   & d   & c   & b   & a^2 & a   & e   \\
    \end{tabular} 
\]
\subsubsection{Rearrangement Theorem}
Sallop gesagt: In jeder Zeile und Spalte einer Multiplikationstabelle kann ein Gruppenelemnt nur einmal auftreten.\\
Mathematisch: In der Sequenz $EA_k, A_2A_k, \cdots , A_h A_k$ kommt jedes Element $A_i$ nur einmal vor.
\subsection{Zyklische Gruppe}
Bei einer zyklischen Gruppe kann jedes Element durch mehrfacher Multiplikation eines Elements reproduziert werden, so dass
sich jede zyklische Gruppe $\symcal{G}$ als 
\begin{equation*}
  \symcal{G} = \{ X, X^2, \ldots, X^n = E \}
\end{equation*}
schreiben lässt, wobei die Ordnung die Periode der zyklischen Gruppe ist (Bsp.: Translationsgruppe eines Kirstalls)
\subsection{Untergruppen und Nebenklassen}
Sei $\symcal{S}  = \{E, S_2, \ldots, S_g \}$ eine Untergruppe der Ordnung $g$ der Gruppe $\symcal{G}$
der Ordnung $h$, dann ist 
\begin{equation*}
  \symcal{S}X = \{ EX, S_2X, \ldots, S_g X\}
\end{equation*}
eine rechte Nebenklasse von $\symcal{S}$ (linke Nebenklasse analog).
Wäre $X \in \symcal{S}$, dann wäre $X\symcal{S}$ wieder $\symcal{S}$ selbst und damit enthält
eine Nebenklasse kein einziges Element der Untergruppe.\\
\subsubsection{Satz: Disjunkheit oder Gleichheit}
Zwei (linke oder rechte) Nebenklassen $X\symcal{S}, Y\symcal{S}$ einer Untergruppe $\symcal{S}$ sind entweder 
disjunkt oder gleich.
\subsubsection{Satz: Index einer Untergruppe}
Die Ordnung einer Untergruppe $\symcal{S}$ von $\symcal{G}$, wobei 
$|\symcal{S}| = g$ und $\symcal{G} = h$ gilt, muss ein ganzzahliger Teiler von $h$ sein, so dass 
\begin{equation*}
  \frac{h}{g} = l \in \symbb{Z}
\end{equation*}
gilt.
Dabei wird $l$ der Index der Untergruppe $\symcal{S}$ in $\symcal{G}$ genannt.
\subsection{Konjugierte Elemente und Klassen}
Zwei Elemente $A, B$ sind zueinander konjugiert, wenn 
\begin{equation*}
  B = X A X^{-1}
\end{equation*}
gilt.
Damit folgt, dass wenn $C$ und $B$ zu $A$ konjugiert sind, dass auch $B$ und $C$ zueinander konjugiert sind.
\subsubsection{Konjugationsklasse}
Alle Elemente einer Gruppe $\symcal{G}$ die zueinander konjugiert sind bilden eine Konjugationsklasse
\begin{equation*}
  \symcal{G} A = \{BAB^{-1} | B \in \symcal{G} \},
\end{equation*}
wobei A ein beliebiges Element der Konjugationsklasse ist.
\subsection{Normalteiler und Faktorgruppen}
\subsubsection{Definition: Normalteiler}
\label{def:normalteiler}
Eine Untergruppe $\symcal{S}$ einer Gruppe $\symcal{G}$, die nur aus kompletten Klassen besteht,
heißt \textbf{Normalteiler} oder \textbf{invariante Untergruppe}.
Mit einer komplette Klasse meint man, dass, wenn $A$ in  $\symcal{S}$ liegt, alle Elemente $XAX^{-1}$ in $\symcal{S}$ liegen,
selbst wenn $X\in\symcal{G}$ nicht in $\symcal{S}$ liegt.
Solche eine Untergruppe heißt invariant, da es unter Konjugation mit einem beliebigen Element 
von $\symcal{G}$ invariant ist.
\subsubsection{Einschub: Komplexe}
Ein Komplex
\begin{equation*}
  \symcal{K} = \{ K_1, \ldots, K_n \} 
\end{equation*}
ist eine Menge von Gruppenelementen unter Vernachlässigung der Reihenfolge.
Eine Multiplikation mit einen beliebigen Element $X$ ist durch $\symcal{K}X = \{ K_1 X, \ldots, K_n X \}$ gegeben.
Die Multiplikation zweier Komplexe $\symcal{K} = \{ K_1, \ldots, K_n \} $ und $\symcal{K}' =  \{ K_1', \ldots, K_m' \} $ ist durch 
$\symcal{K}\symcal{K'} = \{ K_1 K_1', K_2 K_1', \ldots, K_1 K_2', K_1 K_3', \ldots, K_n K_m'  \} $ gegeben.
Doppelte Elemente werden, wie es bei einer Menge üblich ist, nicht mitgezählt.

\subsubsection{Satz: Nebenklasse einer invarianten Untergruppe}
Aus der Definition \ref{def:normalteiler} folgt
\begin{equation*}
  X\symcal{S}X^{-1} = \symcal{S} \iff X\symcal{S} = \symcal{S}X,
\end{equation*}
womit die rechte gleich der linken Nebenklasse einer invarianten Untergruppe ist.
\subsubsection{Definiton: Faktorgruppe}
Eine invariante Untergruppe $\symcal{S}$ einer Gruppe $G$ bildet mit all ihren $l-1$ Nebenklassen eine Faktorgruppe 
\begin{equation*}
  \sfrac{\symcal{G}}{\symcal{S}} = \{S, \symcal{S}X_1, \symcal{S}X_2, \ldots, \symcal{S}X_{l-1} \},
\end{equation*}
wobei die invariante Untergruppe $\symcal{S}$ das Einselement bildet.
Die Ordnung der Faktorgruppe entspricht $\sfrac{|\symcal{G}|}{|\symcal{S}|}$.
\section{Darstellungstheorie}
Wir haben uns ausschließlich mit Matritzendarstellungen beschäftigt.
\subsection{Definition: Darstellung}
Bei einer Darstellung $\symup{\Gamma}$ wird jedem Gruppenelement eine quadratische Matrix zugeordnet:
\begin{equation*}
  \symup{\Gamma}(A) : \quad V \to V
\end{equation*}
mit dem Vektorraum $V$ als Darstellungsraum mit $\text{Dim(V)} = d$ als 
Dimension der Darstellung.
Eine lineare Darstellung $\symup{\Gamma} (A)$ von $\symcal{G}$ ist ein Homomorphismus der Gruppe GL$(V)$
\begin{equation*}
  \symup{\Gamma}(A) \symup{\Gamma}(B) = \symup{\Gamma}(AB), \quad A,B \in \symcal{G}.
\end{equation*} 
Das Einselement wird durch die Einheitsmatrix dargestellt.
\subsection{Definition: Äquivalente Darstellung}
Eine andere Darstellung lässt sich durch eine Ähnlichkeitstransformation gewinnen
\begin{equation*}
  \symup{\Gamma}'(A) = S^{-1} \symup{\Gamma} (A) S \implies \symup{\Gamma}'(A) \symup{\Gamma}'(B) = \symup{\Gamma}'(AB).
\end{equation*}
Die Darstellungen $\symup{\Gamma}$ und $\symup{\Gamma}'$ sind äquivalent.
\subsection{(Ir)reduzibilität}
Die direkte Summe von zwei Darstellungen 
\begin{equation*}
  \symup{\Gamma} (A) = 
  \begin{pmatrix}
    \symup{\Gamma}^1 (A)  & 0           \\
    0             & \symup{\Gamma}^2(A)
  \end{pmatrix}
  , \quad \symup{\Gamma} (A) = \symup{\Gamma}^1 (A)\bigoplus \symup{\Gamma}^2(A)
\end{equation*}
ist eine weiter Form von Redundanz.
Lässt sich eine Darstellung durch eine globale Ähnlichkeitstransformation auf eine Blockdiagonale
bringen, ist sie reduzibel, sonst irreduzibel.
\subsection{Satz: unitäre Darstellungen}
Jede Darstellung lässt sich mit Hilfe einer Ähnlichkeitstransformation auf eine 
unitäre Darstellung abgebildet werden. Vorgehen: Konstruiere hermitische Matrix 
${\mathbf{H} = \sum_i^h \symup{\Gamma}(A_i) \symup{\Gamma} (A_i)^\dagger}$. Dann diese diagonalisieren mit unitärer Trafo 
$\mathbf{d} = \mathbf{U}^{-1} \mathbf{H} \mathbf{U}$. Somit ist
die Darstellung 
\begin{equation*}
  \symup{\Gamma}^{'}(A_j) = \mathbf{d^{-\frac{1}{2}}} \mathbf{U}^{-1} \symup{\Gamma}(A_j) \mathbf{U} \mathbf{d}^{\frac{1}{2}}
\end{equation*}
unitär.
\subsection{Schur'sches Lemma}
Jede Matrix, welche mit allen Matrizen einer irreduziblen Darstellung kommutiert,
muss ein Vielfaches von der Einheitsmatrix (sog. konstante Matrix) sein. 
Wenn somit eine nicht-konstante Matrix mit mindestens einer Matrix einer Darstellung kommutiert, 
ist diese Darstellung reduzibel.
\subsubsection{Alternative Formulierung}
Gegeben seien zwei Darstellungen mit $\text{Dim}(\symup{\Gamma}^1(A_i)) = d_1$ und $\text{Dim}(\symup{\Gamma}^1(A_i)) = d_1$.
Wenn dann mit einer beliebigen Matrix $\mathbf{M}$
\begin{equation*}
  \mathbf{M}\symup{\Gamma}^1(A_1) = \symup{\Gamma}^2(A_i) \mathbf{M}
\end{equation*}
gilt, dann muss (i) bei $d_1 \neq d_2$ $\mathbf{M} = \mathbf{0}$ oder (ii) bei $d_1 = d_2$ entweder 
$\mathbf{M} = \mathbf{0}$ oder $|\mathbf{M}| \neq 0$ gelten.
Aus letzterem folgt $\symup{\Gamma}^1(A_i) = \mathbf{M} \symup{\Gamma}^2(A_i) \mathbf{M}^{-1} $, womit 
die Darstellungen äquivalent sind.
\subsection{Orthogonalitätstheorem}
\label{sub:orth}
Bei Betrachtung \textbf{nicht-äquivalenter}, unitärer, irreduziblen Darstellungen gilt
\begin{equation*}
  \sum_R \symup{\Gamma}^i(R)^*_{\mu \nu}\symup{\Gamma}^j(R)_{\alpha \beta} = \frac{h}{d_i} \delta_{ij}\delta_{\nu \alpha}\delta_{\nu \beta}.
\end{equation*}
Geometrische Interpretation: die Gruppelemente $R = E, A_2, \ldots, A_h$ spannen einen 
$h$-dimensionalen \enquote{Gruppenelement}-Vektorraum auf.
Jeder Vektor in diesem Raum hat drei Indizes, $i, \mu, \nu$.
Diese Vektoren sind orthogonal zueinander.
\subsection{Satz von Burnside}
Aus der geometrischen Interpretation des Orthogonalitätstheorems \ref{sub:orth} folgt 
mit $d_i$ als Dimension der $i$-ten irreduziblen Darstellung der Gruppe $\symcal{G}$ direkt
\begin{equation*}
  \sum_i d_i^2 = |\symcal{G}|,
\end{equation*}
da es zu jeder Darstellung $\symup{\Gamma}^i$ $d_i^2$ verschiedene Vektoren gibt. 
Das heißt, dass in Summe in diesem Vektorraum $\sum_i d_i^2$ verschiedene Vektoren exestieren.
Da in einem $h$-dimensionalen Vektorraum nur maximal $h$ zueinander orthogonale Vektoren exestieren können,
folgt $\sum_i d_i^2 \leq h = |\symcal{G}|$. 
Die eindeutige Gleichheit wird z.B. im Tinkham bewiesen.
\subsection{Definition: Charakter}
Der Charakter einer Darstellung $\symup{\Gamma}^i(R)$ ist die Menge
der $h$ Zahlen $\chi^i(E), \chi^i(A_2), \ldots, \chi^i(A_h)$ mit
\begin{equation*}
  \chi^i(R) = \text{Tr}(\symup{\Gamma}^i(R)) = \sum_j^{d_i} \Gamma^i(R)_{jj}.
\end{equation*}
Die Spur ist unter einer Ähnlichkeitstransformation invariant, so dass äquivalente Darstellungen und 
Elemente innerhalb einer Klasse denselben Charakter besitzen.
Im Folgenden wird der Charakter für die $k$-te Klasse $\symcal{G}_k$ durch $\chi^i(\symcal{G}_k)$ angegeben.
\subsection{Satz: Zeilenorthogonalität}
Wir das Orthogonalitätstheorem genutzt, kann 
\begin{equation*}
  \sum_R \chi^i(R)^* \chi^j(R) = \sum_R \chi^i(\symcal{G}_k)^* \chi^j(\symcal{G}_k) N_k = h \delta_{ij}
\end{equation*}
gezeigt werden, wobei $N_k$ die Anzahl an Elementen in der $k$-ten Klasse ist.
Somit formen die Charaktere der verschiedenen irreduziblen Darstellungen eine Menge von orthogonalen Vektoren 
\begin{equation*}
\chi^i = 
\begin{pmatrix}
  \chi^i(E       )\\
  \chi^i(A_2     )\\
  \ldots          \\
  \chi^i(A_h     )
\end{pmatrix}
\end{equation*}  
im Gruppenelement-Vektorraum aufgespannt durch die Klassen $\symcal{G}_k$.
Da die Anzahl der orthogonalen Vektoren nicht die Dimension des Vektorraums übersteigen kann,
darf die Anzahl der Klassen nicht die Anzahl der irreduziblen Darstellungen übersteigen.
Es gilt sogar
\begin{equation*}
  \text{Anzahl Klassen} = \text{Anzahl irreduzible Darstellungen}.
\end{equation*}
\subsection{Charaktertafel}
In den Zeilen stehen die irreduziblen Darstellungen und in den Spalten die Klassen mit 
der Anzahl an Elementen in der jeweiligen Klasse.
\begin{table}
  \centering
  \caption{Charaktertafel der $S_3$}
  \label{tab:some_data}
  \sisetup{table-format=2.0}
  \begin{tabular}{S[table-format=10.0] S[table-format = 1.0] S S}
  \toprule
   & {$\symcal{G}_1$} & {$3\symcal{G}_2$} & {$2\symcal{G}_3$} \\
  \midrule
  {$\Gamma^1$} & 1 & 1  & 1   \\
  {$\Gamma^2$} & 1 & -1 & 1   \\
  {$\Gamma^3$} & 2 & 0  & -1  \\
  \bottomrule
  \end{tabular}
  \end{table}
\subsection{Satz: Spaltenorthogonalität}
\begin{equation*}
  \sum_i \chi^i(\symcal{G}_k)^* \chi^i(\symcal{G}_l) = \frac{h}{N_k} \delta_{kl}
\end{equation*}
\subsection{Dekomposition von reduziblen Darestellungen}
Bringe die Darstellung erst auf blockdiagonale Form
\begin{equation*}
  \symup{\Gamma} (R) = 
  \begin{pmatrix}
    \symup{\Gamma}^1 (R)  &       &    \\
    & \!\!\!\!\!\! \symup{\Gamma}^2(R) &  \\
    & & \ddots \quad \quad
  \end{pmatrix}
  , \quad \symup{\Gamma} (R) = \symup{\Gamma}^1 (R)\bigoplus \symup{\Gamma}^2(R) \bigoplus \cdots
\end{equation*}
mit den irreduziblen Darstellungen auf der Diagonale.
Somit ist die Spur der reduziblen Darstellung die Summe der Spuren der irreduziblen Darstellungen
\begin{equation*}
  \chi_{\text{red}} (R) = \sum_i a_i \chi_{\text{irred}}^i(R),
\end{equation*}
wobei der Koeffizient $a_i$ angibt, wie oft die $i$-te irreduzible Darstellung auf der 
Diagonale vorkommt.
Durch Anwendung des Orthogonalitätstheorems \ref{sub:orth}, sind die Koeffizienten eindeutig 
durch den Charakter der reduziblen Darstellung bestimmt.
\begin{equation*}
  a_i = \frac{1}{h} \sum_k N_k \chi_{\text{irred}}^i (\symcal{G}_k)^* \chi_{\text{red}} (\symcal{G}_k) . \label{eqn:decomp}
\end{equation*}
\subsection{Reguläre Darstellung}
Hier schiebt man einfach die Elemente in der Multiplikationstabelle so rum, so dass 
auf der Hauptdiagonalen das Einselement liegt. 
Wenn man nun die reguläre Darstellung eines Elements bestimmen möchte, schaut man in der 
Multiplikationstabelle, wo dieses Element als Resultät der Multiplikationen steht.
Damit bekommt die Matrix der regulären Darstellung eine $1$ als Eintrag an dieser Stelle.
\[
    \begin{tabular}{>{$}l<{$}|*{6}{>{$}l<{$}}}
    ~       & E   & A & B & C & D & F   \\
    \hline\vrule height 12pt width 0pt
    E       & E   & A & B & C & D & F\\
    A^{-1}  & A   & E & D & F & B & C\\
    B^{-1}  & B   & F & E & D & C & A\\
    C^{-1}  & C   & D & F & E & A & B\\
    D^{-1}  & F   & B & C & A & E & D\\
    F^{-1}  & D   & C & A & B & F & E\\
    \end{tabular} 
\]
\begin{equation*}
  \Gamma_\text{reg}(A)
  \begin{pmatrix}
    0 & 1 & 0 & 0 & 0 & 0 \\
    1 & 0 & 0 & 0 & 0 & 0 \\
    0 & 0 & 0 & 0 & 0 & 1 \\
    0 & 0 & 0 & 1 & 1 & 0 \\
    0 & 0 & 0 & 0 & 0 & 0 \\
    0 & 0 & 1 & 0 & 0 & 0 
  \end{pmatrix}
\end{equation*}
Aus \begin{equation*}
  \chi_\text{reg} \cdot \chi_\text{irred}^i = 
  \begin{pmatrix}
    h \\
    0 \\
    \cdots
  \end{pmatrix}
  \cdot 
  \begin{pmatrix}
    d_i \\
    x   \\
    \cdots
  \end{pmatrix}
  = h d_i 
\end{equation*}
folgt, dass die reguläre Darstellung \textbf{jede} irreduzible Darstellung genau $d_i$ mal entält (s. Gl. \eqref{eqn:decomp}).
\section{Symmetrieoperationen in der Quantenmechanik}
Ausgangspunkt ist die Schrödingergleichung 
\begin{equation*}
  \hat {H} \Psi_n = E_n^j \Psi_n.
\end{equation*}
Dabei gibt j den Grad der Entartung an.
Ziel ist es nun einen Zusammenhang zwischen den Symmetrieoperationen und der Entartung der Energieniveaus 
zu finden.
\subsection{Wirkung der Symmetrieoperationen auf Wellenfunktionen}
Sei $\hat{P}_R$ eine Symmetrieoperationen, somit wirkt diese auf eine Wellenfunktionen gemäß
\begin{equation*}
  \hat{P}_R \Psi (\vec{r}) = \Psi(\mathbf{R}^{-1}\vec{r}).
\end{equation*}
Somit kann entweder der Vektor um einen Winkel im mathematisch positiven Drehsinn oder 
das Koordinatensystem um diese Winkel im mathematisch negativen Drehsinn gedreht werden.
Die Symmetrieoperatoren bilden eine Gruppe, die isomoprh zu der Gruppe der 
Koordinatentransformationen ist.
\subsection{Symmetrie des Hamiltonoperators}
\label{sub:symmhamil}
Wenn das System und damit auch der Hamiltonian invariant unter einer Symmetrieoperationen ist, vertauscht 
der Hamiltonian mit dem Symmetrieoperator und somit gilt 
\begin{equation*}
  \hat{H} \Psi = E_n \Psi \iff 
  \hat{P}_R \hat{H} \Psi = E_n \hat{P}_R \Psi \iff 
  \hat{H} \hat{P}_R \Psi = \hat{H} \Psi' = E_n \Psi',
\end{equation*}
d.h. bei Symmetrien haben verschiedene Zustände die selbe Energie, womit Entartung vorliegt.
\subsection{Die Gruppe der Schrödingergleichung}
\label{sub:schrdgroup}
Sei nun eine Energieniveau $E_n$ $d_n$-fach entartet.
Dann wähle $d_n$ orthonormale Eigenfunktionen, welche zu $E_n$ gehören.
Diese $d_n$ Eigenfunktionen spannen den entarteten Unterraum $\symcal{V} \in \symcal{H}$ 
von dem gesamten Hilbertraum $\symcal{H}$ auf, welcher invariant unter den Symmetrieoperationen, welche 
lineare Abbildungen 
\begin{equation*}
  \hat{P}_R: \quad \symcal{V} \to \symcal{V}
\end{equation*}
sind, ist.
Die Darstellungen sind irreduzibel, da in dem Raum $\symcal{V}$ kein invarianter Unterraum exestiert.
\subsection{Bestimmung einer Darstellung der Symmetriegruppe}
Nach Abschnitt \ref{sub:schrdgroup} erhält man durch Anwendung der Symmetrieoperation auf eine 
Eigenfunktion des entarteten Unterraums eine Linearkombination aller Eigenfunktionen
\begin{equation*}
  \hat{P}_R \Psi_\nu^n = \sum_\kappa \Gamma^n(R)_{\kappa \nu} \Psi_\kappa^n
\end{equation*}
Die Matrixen $\Gamma^n(R)$ bilden dann eine irreduzible Darstellung zu der Symmetriegruppe, unter welcher 
der entartete Unterraum mit dem Energiveau $E_n$ invariant ist.
\subsection{Unitarität der Darstellung}
Nur wenn die Eigenfunktionen orthornmiert gewählt werden, sind die Darstellungen unitär.
\subsection{Entartungsgrad}
\begin{equation*}
  \text{Entartungsgrad} = \text{Dimension der irreduziblen Darstellung}
\end{equation*}
\section{Liegruppen}
\begin{enumerate}
\item erfüllen Gruppenaxiome 
\item besitzen eine analytische Mannigfaltigkeit
\item Abstandsbegriff (bilden topologischen Raum, metrischen Raum)
\end{enumerate}
\subsection{Definiton: Liegruppe}
Eine Gruppe $\symcal{G}$ wird Liegruppe genannt, wenn
\begin{enumerate}
  \item $\symcal{G}$ besitzt mindestens eine endliche Darstellung $\Gamma(T)$ mit $T \in \symcal{G}$ der Dimension 
  $d$. 
  Definiere einen Abstand 
    \begin{equation*}
      d(T,T') = \sqrt{\sum_{\substack{j=1}{k=1}} \left | \Gamma(T)_{jk} - \Gamma(T')_{jk} \right |^2  } = d(T',T) > 0
    \end{equation*}
    und somit eine Metrik.
  \item Jedes Element $T$ kann durch $n$ reelle Parameter $\theta = (\theta_1, \ldots, \theta_n)$ angegeben werden.
    $n$ ist dann die Dimension der Liegruppe und die minimale Anzahl an Elementen.
  \item Nähe zum Einselement. Sei $\eta$ vorgegeben, dann gehört zu jedem Punkt $\theta =  (\theta_1, \ldots, \theta_n)$,
    für den $\sum_i^n \theta_i^2 < \eta^2$ gilt, ein Element $T \in \symcal{G}$.
  \item Die Darstellungen $\Gamma(T(\theta))$ sind in den Parametern analytisch und somit in eine Potenzreihe entwickelbar.
\end{enumerate}
Die Gruppenelemente $T(\theta)$ sind \enquote{smooth} bzgl. den Parametern. 
Das bedeutet, dass die wenn die Gruppenelemente nah beieinander sind, dass die Parameter ebenfalls nah beieinander sind.
Ebenfalls soll das Einselement bei $\theta = 0$ liegen:
\begin{equation*}
  T(\theta)\big |_{\theta = 0} = E \iff \Gamma(T(\theta)) \big |_{\theta = 0} = \mathbb{1}
\end{equation*}
\subsection{Erzeugenden}
Die $n$ Erzeugenden/Generatoren einer $n$-dimensionalen Liealgebra sind gemäß 
\begin{equation*}
  \frac{1}{i} \frac{\partial}{\partial \theta_j} \Gamma(T(\theta)) \biggr |_{\theta = 0} = t_j \label{eqn:gen}
\end{equation*}
definiert. 
Der Faktor $\sfrac{1}{i}$ ist Konvention.
Wenn man die Darstellungen bis zur ersten Ordnung entwickelt und fordert, dass die Darstellungen 
unitär sind, folgt dass die Generatoren hermitesch sein müssen. 
\begin{align*}
  \Gamma(T(\theta)) &= \symbb{1} + \frac{\partial \Gamma(T(\theta))}{\partial \theta_a} \biggr |_{\theta = 0} \theta_a + \symbfcal{O}(\theta^2) 
  = \mathbb{1} + i t_a \theta_a + \symbfcal{O}(\theta^2) \\
  \implies \Gamma(T(\theta))\Gamma(T(\theta))^{\dagger}  &\approx \mathbb{1} + i \theta_a (t_a - t_a^{\dagger}) \stackrel{!}{=} \mathbb{1} \iff t_a = t_a^{\dagger}
\end{align*}
\subsection{Strukturkonstanten}
Ausgang ist die Gruppeneigenschaft. 
Somit muss mit $T(\theta), \; T(\theta') \in \symcal{G}$ $T(\theta)T(\theta') = T(f(\theta, \theta'))$ gelten.
Es gilt ebenfalls $\theta = \theta_1, \ldots, \theta_n, \; f = f_1, \ldots, f_n$ mit n als Dimension der Gruppe.
Bestimme nun $f(\theta, \theta')$ und entwickle $f$ in die zweite Ordnung.
Dann nochmal die Darstellungen in erste Ordnung entwickeln und dann Koeffizientenvergleich machen, womit 
die Bedingung für eine Liealgebra
\begin{equation*}
  [t_a, t_b] = i \sum_c^n f_{ab}^ct_c
\end{equation*}
folgt. 
Die $f_{ab}^c$ als sind Strukturkonstanten,
welche extrem wichtig sind, da diese die gesamte Gruppenmultiplikation zusammenfassen.
\subsection{Exponentielle Parameterisierung}
Wenn man nun etwas von der Eins weggeht, lässt sich die Darstellung durch 
ein Entwicklung in den Parametern darstellen:
\begin{equation*}
  \Gamma(\symup{d}\theta) = \mathbb{1} + i \sum_a^n t_a \symup{d}\theta_a 
\end{equation*}
Wir können $\symup{d}\theta_a$ auch einfach durch $\symup{d}\theta_a = \frac{\theta}{k}$ darstellen und $k$ gegeben
Unendlich laufen lassen.
Damit lässt sich die Darstellung
\begin{equation*}
  \Gamma(\theta) = \lim_{k \to \infty} (\mathbb{1} + i \sum_a^n t_a \frac{\theta_a}{k})^k = \symup{e}^{i\sum_a^n t_a \theta_a}
  = \symup{e}^{i\vec{t} \cdot \vec{\theta}}
\end{equation*}
finden.
\subsection{Lie-Klammern}
Die Erzeugenden erfüllen die Axiome einer Liealgebra, d.h. für die Generatoren $t_a \in \symcal{V}$ ist 
zusätzlich eine Verknüpfung, die Lie-Klammer, 
\begin{align*}
  \symcal{V} * \symcal{V} &\to \symcal{V} \\
  x,y \in \symcal{V}      &\to [x,y] \in \symcal{V}
\end{align*}
definieren, welche die Eigenschaften 
\begin{enumerate}
  \item Biliniarität: $[\alpha x + \beta y, z] = \alpha [x,z] + \beta [y,z]$
  \item $[x,x] = 0$
  \item Jacobi-Identität (zyklische Vertauschung): $[x, [y,z]] + [z,[x,y]] + [y,[z,x]] = 0$
\end{enumerate}
erfüllen.
\subsection{SO(3)}
S für speziell (Determinante ist eins), O für orthogonal und drei für $3 \times 3$ Darstellungsmatrizen.
Die SO(3) ist eigentlich einfach nur  die Gruppe der Drehungen im $\mathbb{R}^3$.
Generatoren $J_i$ gemäß Gleichung \eqref{eqn:gen} bestimmen:
\begin{equation*}
  J_1 = 
  \begin{pmatrix}
    0 & 0 & 0   \\
    0 & 0 & -i  \\
    0 & i & 0 
  \end{pmatrix}, 
  J_2 = 
  \begin{pmatrix}
    0 & 0   & i \\
    0 & 0   & 0 \\
    -i & 0  & 0 
  \end{pmatrix}
  J_3= 
  \begin{pmatrix}
    0 & -i  & 0 \\
    i & 0   & 0 \\
    0 & 0   & 0 
  \end{pmatrix}
\end{equation*}
Die Kommutatorrelation ist 
\begin{equation*}
  [J_i, J_j] = i \sum_k \epsilon_{ijk} J_k 
\end{equation*}
mit dem epsilon-Tensor als Strukturkonstanten und somit bilden die Generatoren $J_i$ eine Liealgebra.
Die Gruppenmannigfaltigkeit wird durch  Drehungen im $\mathbb{R}^3$ auf einer Vollkugel mit dem Radius $\pi$ 
um den Vektor $\hat{k} = \sfrac{\vec{k}}{|\vec{k}|} = (k_1, k_2, k_3)^T$ mit $|\vec{k}| = \theta$ als
Drehwinkel beschrieben.
Nach der Rodrigues-Formel gilt 
\begin{equation*}
  \Gamma_{ij} = (1-\cos(\theta)) k_i k_j + \cos(\theta) \delta_{ij} + \sin(\theta) \sum_m \epsilon_{imj}k_m.
\end{equation*}
Es gilt außerdem $\Gamma(\hat{k}) = \Gamma(-\hat{k})$.
\subsection{SU(2)}

\end{document}