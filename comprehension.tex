\documentclass[
  captions=tableheading,  % Tabellenüberschriften
  titlepage=firstiscover, % Titelseite ist Deckblatt
]{scrartcl}

% Paket float verbessern
\usepackage{scrhack}

% Warnung, falls nochmal kompiliert werden muss
\usepackage[aux]{rerunfilecheck}

% unverzichtbare Mathe-Befehle
\usepackage{amsmath}
% viele Mathe-Symbole
\usepackage{amssymb}
% Erweiterungen für amsmath
\usepackage{mathtools}

\usepackage{amsthm}


% Fonteinstellungen
\usepackage{fontspec}
% Latin Modern Fonts werden automatisch geladen
% Alternativ zum Beispiel:
%\setromanfont{Libertinus Serif}
%\setsansfont{Libertinus Sans}
%\setmonofont{Libertinus Mono}

% Wenn man andere Schriftarten gesetzt hat,
% sollte man das Seiten-Layout neu berechnen lassen
\recalctypearea{}

% deutsche Spracheinstellungen
\usepackage[ngerman]{babel}

\usepackage{thmtools}

\usepackage[
  math-style=ISO,    % ┐
  bold-style=ISO,    % │
  sans-style=italic, % │ ISO-Standard folgen
  nabla=upright,     % │
  partial=upright,   % ┘
  warnings-off={           % ┐
    mathtools-colon,       % │ unnötige Warnungen ausschalten
    mathtools-overbracket, % │
  },                       % ┘
]{unicode-math}

% traditionelle Fonts für Mathematik
\setmathfont{Latin Modern Math}
% Alternativ zum Beispiel:
%\setmathfont{Libertinus Math}

\setmathfont{XITS Math}[range={scr, bfscr}]
\setmathfont{XITS Math}[range={cal, bfcal}, StylisticSet=1]

% Zahlen und Einheiten
\usepackage[
  locale=DE,                   % deutsche Einstellungen
  separate-uncertainty=true,   % immer Unsicherheit mit \pm
  per-mode=symbol-or-fraction, % / in inline math, fraction in display math
]{siunitx}

% chemische Formeln
\usepackage[
  version=4,
  math-greek=default, % ┐ mit unicode-math zusammenarbeiten
  text-greek=default, % ┘
]{mhchem}

% richtige Anführungszeichen
\usepackage[autostyle]{csquotes}

% schöne Brüche im Text
\usepackage{xfrac}

% Standardplatzierung für Floats einstellen
\usepackage{float}
\floatplacement{figure}{htbp}
\floatplacement{table}{htbp}

% Floats innerhalb einer Section halten
\usepackage[
  section, % Floats innerhalb der Section halten
  below,   % unterhalb der Section aber auf der selben Seite ist ok
]{placeins}

% Seite drehen für breite Tabellen: landscape Umgebung
\usepackage{pdflscape}

% Captions schöner machen.
\usepackage[
  labelfont=bf,        % Tabelle x: Abbildung y: ist jetzt fett
  font=small,          % Schrift etwas kleiner als Dokument
  width=0.9\textwidth, % maximale Breite einer Caption schmaler
]{caption}
% subfigure, subtable, subref
\usepackage{subcaption}

% Grafiken können eingebunden werden
\usepackage{graphicx}

% schöne Tabellen
\usepackage{booktabs}

% Verbesserungen am Schriftbild
\usepackage{microtype}


\usepackage{listings}

% Hyperlinks im Dokument
\usepackage[
  german,
  unicode,        % Unicode in PDF-Attributen erlauben
  pdfusetitle,    % Titel, Autoren und Datum als PDF-Attribute
  pdfcreator={},  % ┐ PDF-Attribute säubern
  pdfproducer={}, % ┘
]{hyperref}
% erweiterte Bookmarks im PDF
\usepackage{bookmark}

% Trennung von Wörtern mit Strichen
\usepackage[shortcuts]{extdash}

\usepackage{cleveref}

\author{%
  Yanick Sebastian Kind\\%
  \href{mailto:yanick.kind@udo.edu}{yanick.kind@udo.edu}%
}

\title{Zusammenfassung zur Vorlesung Gruppentheorie I}
\date{03.03.2023}

\begin{document}

\maketitle

\tableofcontents

\theoremstyle{remark}
\newtheorem{df}{Definition}
\listoftheorems


\newpage
\section{Abstrakte Gruppentheorie}
\subsection{Definition: Gruppe}
  Eine Menge $\symcal{G} = \{A_2,A_3,...\}$ bildet eine Gruppe, wenn mit einer Gruppenverknüpfung $*$ folgende vier Eigenschaften erfüllt sind:
  \begin{enumerate}
    \item \textbf{Abgeschlossenheit}: Mit $A_i, A_j \in \symcal{G}$ folgt $A_i * A_j = A_k \in \symcal{G}$, d.h. die Verknüpfung zweier 
    Elemente ergibt wieder ein Element der Gruppe.
    \item \textbf{Assoziativität}: Es gilt mit $A_i, A_j, A_k \in \symcal{G}$, dass $(A_i * A_j) * A_k = A_i * (A_j*A_k)$.
    \item \textbf{Neutrale Element}: Es exestiert ein eindeutiges Element $E\in \symcal{G}$ mit $E * A_i = A_i * E  = A_i$.
    \item \textbf{Inverse Element}: Zu jedem Element $A_i \in \symcal{G}$ exestiert ein eindeutiges inverses Element $A_i^{-1}$,
      so dass $A_i^{-1} * A_i = A_i * A_i^{-1} = E$ gilt.
  \end{enumerate}
\subsubsection{endliche Gruppe}
Eine Gruppe mit einer endlichen Anzahl an Elementen heißt endliche Gruppe.
Eine Gruppe $\symcal{G} = \{ E, A_2, \ldots, A_h \} $ ist eine endliche Gruppe der Ordnung $h$.\\
\footnote{Im Folgenden wird das Symbol der Verknüpfung und die Angabe, dass ein Element ein Element
einer Gruppe ist, weggelassen, sofern es eindeutig ist.}
\subsection{Multiplikationstabelle}
Die Multiplikationstabelle gibt einfach an, welche Verknüpfungen welches Gruppenelement ergeben.
Bsp. Symmetrische Gruppe $S_3$:
\[
    \begin{tabular}{>{$}l<{$}|*{6}{>{$}l<{$}}}
    ~   & e   & a   & a^2 & b   & c   & d   \\
    \hline\vrule height 12pt width 0pt
    e   & e   & a   & a^2 & b   & c   & d   \\
    a   & a   & a^2 & e   & c   & d   & b   \\
    a^2 & a^2 & e   & a   & d   & b   & c   \\
    b   & b   & d   & c   & e   & a^2 & a   \\
    c   & c   & b   & d   & a   & e   & a^2 \\
    d   & d   & c   & b   & a^2 & a   & e   \\
    \end{tabular} 
\]
\subsubsection{Rearrangement Theorem}
Sallop gesacht: In jeder Zeile und Spalte einer Multiplikationstabelle kann ein Gruppenelemnt nur einmal auftreten.\\
Mathematisch: In der Sequenz $EA_k, A_2A_k, \cdots , A_h A_k$ kommt jedes Element $A_i$ nur einmal vor.
\subsection{Zyklische Gruppe}
Bei einer zyklischen Gruppe kann jedes Element durch mehrfacher Multiplikation eines Elements reproduziert werden, so dass
sich jede zyklische Gruppe $\symcal{G}$ als 
\begin{equation*}
  \symcal{G} = \{ X, X^2, \ldots, X^n = E \}
\end{equation*}
schreiben lässt, wobei die Ordnung die Periode der zyklischen Gruppe ist (Bsp.: Translationsgruppe eines Kirstalls)
\subsection{Untergruppen und Nebenklassen}
Sei $\symcal{S}  = \{E, S_2, \ldots, S_g \}$ eine Untergruppe der Ordnung $h$ der Gruppe $\symcal{G}$
der Ordnung $h$, dann ist 
\begin{equation*}
  X\symcal{S} = \{ EX, S_2X, \ldots, S_g X\}
\end{equation*}
eine rechte Nebenklasse von $\symcal{S}$ (linke Nebenklasse analog).
Wäre $X \in \symcal{S}$, dann wäre $X\symcal{S}$ wieder $\symcal{S}$ selbst und damit enthält
eine Nebenklasse kein einziges Element der Untergruppe.\\
\subsubsection{Satz: Disjunkheit oder Gleichheit}
Zwei (linke oder rechte) Nebenklassen $X\symcal{S}, Y\symcal{S}$ einer Untergruppe $\symcal{S}$ sind entweder 
disjunkt oder gleich.
\subsection{Satz: Index einer Untergruppe}


\end{document}
